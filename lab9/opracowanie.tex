\documentclass{article}

\usepackage{graphicx}
\usepackage{systeme}
\usepackage[T1]{fontenc}


\title{Lab 9}
\author{Filip Jędrzejewski}

\begin{document}
	\maketitle
	
	\section*{Zadanie 1}
	
	\subsection*{Opis problemu}

	Celem zadania było przedstawienie każdego z poniższych równań różniczkowych zwyczajnych jako równoważny układ równań pierwszego rzędu.

	\subsection*{Równanie Van der Pol’a}

	\begin{equation}
		y'' = y' (1-y^2)-y
	\end{equation}

	Zapiszmy:

	\begin{equation}
		y_1 = y
	\end{equation}

	\begin{equation}
		y_2 = y'
	\end{equation}
	
	Zatem:

	\begin{equation}
		\left\{\begin{array}{rcl}
			y_1'&=&y_2\\
			y_2'&=&y_2(1-y_1^2)-y_1\\
			\end{array} \right.
	\end{equation}


	\subsection*{Równanie Blasiusa}

	\begin{equation}
		y''' = -yy''
	\end{equation}

	Zapiszmy:

	\begin{equation}
		y_1 = y
	\end{equation}

	\begin{equation}
		y_2 = y'
	\end{equation}

	\begin{equation}
		y_3 = y''
	\end{equation}


	\newpage

	Zatem:

	\begin{equation}
		\left\{\begin{array}{rcl}
			y_1'&=&y_2\\
			y_2'&=&y_3\\
			y_3'&=&-y_1y_3
			\end{array} \right.
	\end{equation}


	\subsection*{Prawo powszechnego ciążenia dla problemu dwóch ciał}

	\begin{equation}
		\left\{\begin{array}{rcl}
			y_1''&=&-GM \frac{y_1}{(y_1^2+y_2^2)^{\frac{3}{2}}}\\
			y_2''&=&-GM \frac{y_2}{(y_1^2+y_2^2)^{\frac{3}{2}}}
			\end{array} \right.
	\end{equation}

	Zapiszmy:

	\begin{equation}
		x_1 = y_1
	\end{equation}

	\begin{equation}
		x_2 = y_1'
	\end{equation}

	\begin{equation}
		x_{I} = y_2
	\end{equation}

	\begin{equation}
		x_{II} = y_2'
	\end{equation}

	\begin{equation}
		R = (y_1^2+y_2^2)^{\frac{1}{2}} = (x_1^2+x_I^2)^{\frac{1}{2}}
	\end{equation}

	Zatem:

	\begin{equation}
		\left\{\begin{array}{rcl}
			x_1'&=&x_2\\
			x_I'&=&x_{II}\\
			x_2'&=&-GM \cdot x_1 R^{-3}\\
			x_{II}'&=&-GM  \cdot x_I R^{-3}
			\end{array} \right.
	\end{equation}


	\newpage


	\section*{Zadanie 2}
	
	\subsection*{Opis problemu}

	Dane jest równanie różniczkowe zwyczajne:

	\begin{equation}
		y' = -5y
	\end{equation}

	z warunkiem początkowym $y(0) = 1$. Równanie rozwiązujemy numerycznie z krokiem $h = 0,5$ . 

	\subsection*{Czy rozwiązania powyższego równania są stabilne?}


	
	
	
	
	
	
\end{document}