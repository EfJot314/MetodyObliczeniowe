\documentclass{article}

\usepackage{graphicx}

\title{Lab 6}
\author{Filip Jędrzejewski}

\begin{document}
	% \maketitle
	
	\section*{Zadanie 1}
	
	\subsection*{Opis problemu}
	
	Celem zadania było wyznaczanie wartości $\pi$ wykorzystując następujący wzór:

	\begin{equation}
		\frac{4}{1 + x^2} = \pi
	\end{equation}

	Całkę po lewej stronie równosci wyznaczano numerycznie otrzymując przybliżone wartości $\pi$, 
	na podstawie których badano różne metody całkowania numerycznego.
	
	
	
	
	
	
	
	
	
	
	
	
	
	
	
	
	
	
	
	
	
	
	
	
\end{document}